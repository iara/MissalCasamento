\begin{center}
	\textbf{COMUNHÃO}
	\\\textit{Participação no Sacrifício}
	\\
	\textbf{PATER NOSTER}
\end{center}

\begin{multicols}{2}
	\noindent ORÉMUS. Præcéptis salutáribus móniti, et divína institutióne formáti, audémus dícere:
	\\
	\\ Pater noster, qui es in cælis: sanctificétur nomen tuum: advéniat regnum tuum: fiat volúntas tua, sicut in cælo, et in terra. Panem nostrum quotidiánum da nobis hódie: et dimítte nobis débita nostra, sicut et nos dimíttimus debitóribus nostris. Et ne nos indúcas in tentatiónem,
	\\ \textbf{R. Sed líbera nos a malo. }
	\\ OREMOS. Advertidos pelos preceitos do Salvador e instruídos pelos divinos ensinamentos, ousamos dizer:
	\\Pai nosso, que estais no céu, santificado seja o vosso nome, venha a nós o vosso reino, seja feita a vossa vontade, assim na terra, como no céu. O pão nosso de cada dia nos dai hoje, perdoai-nos as nossas dívidas, assim como nós perdoamos aos nossos devedores, e não nos deixeis cair em tentação.
	\\ R. Mas livrai-nos do mal.
\end{multicols}
\begin{center}
	\textbf{BÊNÇÃO NUPCIAL}
\end{center}
\begin{flushleft}
	Depois do Pater, antes de dizer \textit{Libera no}s, o sacerdote, voltado para os esposos, que estão ajoelhados, reza sobre eles as orações seguintes:
\end{flushleft}
\begin{multicols}{2}
	\noindent Propitiáre, Dómine, supplicatiónibus nostris, et institútis tuis, quibus propagatiónem humáni géneris ordinásti, benígnus assíste: ut, quod te auctóre júngitur, te auxiliánte servétur. Per Dóminum.
	Oremos. 
	\\
	\\
	\\
	\\
	\\ Senhor, sede favorável às nossas orações, assisti à instituição do Matrimônio pelo qual regulastes o crescimento do gênero humano, a fim de que esta união, da qual sois o autor, se conserve pela vossa graça. Por Jesus Cristo, vosso Filho e nosso Senhor, que convosco vive e reina na unidade do Espírito Santo, Deus, por todos os séculos dos séculos. Amém.
\end{multicols}
\newpage
\begin{multicols}{2}
	\noindent Deus, qui potestáte virtútis tuae de níhilo cuncta fecísti: qui, dispósitis universitátis exórdiis, hómini, ad imáginem Dei facto, ídeo inseparábile mulíeris adjutórium condidísti, ut femíneo córpori de viríli dares carne princípium, docens quod ex uno placuísset instítui, numquam licére disjúngi: Deus, qui tam excellénti mystério conjugálem cópulam consecrásti, ut Christi et Ecclésiae sacraméntum praesignáres in foédere nuptiárum: Deus, per quem múlier júngitur viro, et socíetas principáliter ordináta ea benedictióne donátur, quae sola nec per originális peccáti poenam, nec per dilúvii est abláta senténtiam: réspice propítius super hanc fámulam tuam, quae maritáli jungénda consórtio, tua se éxpetit protectióne muníri: sit in ea jugum dilectiónis et pacis: fidélis et casta nubat in Christo, imitatríxque sanctárum permáneat feminárum: sit amábilis viro suo, ut Rachel : sápiens, ut Rebécca: longaéva et fidélis, ut Sara: nihil in ea ex áctibus suis ille auctor praevaricatiónis usúrpet: nexa fídei, mandatísque permáneat: uni thoro juncta, contáctus illícitos fúgiat: múniat infirmitátem suam róbore disciplínae: sit verecúndia gravis, pudóre venerábilis, doctrínis caeléstibus erudíta: sit foecúnda in sóbole, sit probáta et ínnocens: 
	\\
	\\
	\\ Ó Deus, que por vossa onipotência criastes do nada todas as coisas, pusestes desde o início a harmonia no mundo, fizestes o homem à vossa imagem e lhe destes na mulher uma companheira inseparável; tirando da carne do homem o corpo da mulher, ensinastes que não é jamais permitido separar o que quisestes fazer sair de um só ser.Ó Deus, que por um tão grande mistério consagrastes a união conjugal, fazendo que ela prefigure as núpcias de Cristo e da Igreja. Ó Deus, que unis a mulher ao homem e dais a esta união, estabelecida desde o princípio, a única bênção que não foi abolida nem pelo castigo do pecado original, nem pela condenação do dilúvio. Olhai com bondade esta vossa serva, que pede a vossa proteção, no momento em que une a própria sorte à de seu esposo pelo Matrimônio. Seja-lhe ela unida pelo jugo do amor e da paz. Torne-se, em Cristo, uma esposa fiel e casta, a exemplo das santas mulheres, amável a seu marido como Raquel, prudente como Rebeca, fiel durante uma longa vida como Sara. Nada em sua vida dê ocasião ao demônio, autor do pecado. Fique sempre fiel à fé e aos mandamentos. Unida a seu marido, evite toda relação ilegítima. Sustente a sua fraqueza na disciplina. Sua discrição lhe mereça a estima, seu pudor inspire respeito e seja instruída nas coisas de Deus. Tenha ela a maternidade fecunda. Seja pura
\end{multicols}
\begin{multicols}{2}
	\noindent et ad Beatórum réquiem atque ad caeléstia regna pervéniat: et vídeant ambo fílios filiórum suórum usque in tértiam et quartam generatiónem, et ad optátam pervéniant senectútem. Per eúmdem Dóminum nostrum Jesum Christum.
	\\e irrepreensível e chegue ao repouso dos eleitos no Reino do céu. E vejam ambos os filhos dos seus filhos até à terceira e quarta geração, atingindo uma feliz velhice. Por Cristo, nosso Senhor. Amém.
\end{multicols}
\begin{flushleft}
	\textit{O celebrante diz \textbf{Amen} em voz baixa, e continua:}
\end{flushleft}
\begin{multicols}{2}
	\noindent Líbera nos, quáesumus, Dómine, ab ómnibus malis, prætéritis, præséntibus et futúris: et intercedénte beáta et gloriósa semper Vírgine Dei Genitríce María, cum beátis Apóstolis tuis Petro et Paulo, atque Andréa, et ómnibus Sanctis, da propítius pacem in diébus nostris; ut, ope misericórdiæ tuæ adjúti, et a peccáto simus semper líberi et ab omni perturbatióne secúri. Per eúmdem Dóminum nostrum Jesum Christum, Fílium tuum. Qui tecum vivit et regnat in unitáte Spíritus Sancti Deus, \textit{Per ómnia saécula saeculórum.}
	\\ \textbf{R. Amen. }
	\\
	\\ Livrai-nos, Senhor, de todos os males, passados, presentes e futuros, e, pela intercessão da bem-aventurada e gloriosa sempre Virgem Maria, Mãe de Deus, dos bem-aventurados Apóstolos Pedro e Paulo e André, e de todos os Santos, concedei-nos propício a paz em nossos dias; de modo que, ajudados com os auxílios da vossa misericórdia, sejamos sempre livres do pecado e assegurados contra toda a perturbação. Por nosso Senhor Jesus Cristo, vosso Filho, que, sendo Deus, convosco vive e reina na unidade do Espírito Santo, \textit{Por todos os séculos dos séculos. }
	\\ R. Amém. 
\end{multicols}
\begin{center}
	\textbf{FRAÇÃO DA HÓSTIA}
\end{center}
\begin{flushleft}
	O celebrante parte a Hóstia ao meio, de uma das partes tira um pequeno fragmento que deita no preciosíssimo Sangue, traçando antes, com ele, sobre o Cálice, três vezes, o sinal da cruz, e
	dizendo:
\end{flushleft}
\begin{multicols}{2}
	\noindent Pax $\dag$ Dómini $\dag$ sit semper vobís$\dag$cum. 
	\\ \textbf{R. Et cum spíritu tuo. }
	\\ A paz $\dag$ do Senhor $\dag$ seja sempre com$\dag$vosco.
	\\ R. E com o teu espírito.
\end{multicols}
\begin{multicols}{2}
	\noindent Hæc commíxtio et consecrátio Córporis et
	Sánguinis Dómini nostri Jesu Christi fiat
	accipiéntibus nobis in vitam ætérnam. Amen.
	\\
	\\ Que esta mistura sacramental do Corpo e do
	Sangue de Nosso Senhor Jesus Cristo, seja
	para nós que os vamos receber, penhor da
	vida eterna. Amém.
\end{multicols}
\begin{center}
	\textbf{AGNUS DEI}
\end{center}
\begin{multicols}{2}
	\noindent Agnus Dei, qui tollis peccáta mundi, 
	\\ \textbf{R. Miserére nobis}. 
	\\
	\\Agnus Dei, qui tollis peccáta mundi, 
	\\ \textbf{R. Miserére nobis. }
	\\
	\\Agnus Dei, qui tollis peccáta mundi, 
	\\ \textbf{R. Dona nobis pacem. }
	\\
	\\Cordeiro de Deus, que tirais os pecados do mundo, 
	\\R. Tende misericórdia de nós. 
	\\Cordeiro de Deus, que tirais os pecados do mundo, 
	\\R. Tende misericórdia de nós. 
	\\Cordeiro de Deus, que tirais os pecados do mundo, 
	\\R. Dai-nos a paz.
\end{multicols}
\begin{flushleft}
	\textit{Inclinado, recita a oração seguinte, pela paz da Igreja.}
\end{flushleft}
\begin{multicols}{2}
	\noindent Dómine Jesu Christe, qui dixísti Apóstolis tuis: Pacem relínquo vobis, pacem meam do vobis: ne respícias peccáta mea, sed fidem Ecclésiæ tua; eámque secúndum voluntátem tuam pacificáre et coadunáre dignéris: qui vivis et regnas Deus per ómnia sæcula sæculórum. Amen. 
	\\
	\\ Senhor Jesus Cristo, que dissestes aos vossos Apóstolos: "Deixo-Vos a paz, dou-vos a minha paz", não olheis para os meus pecados, mas para a fé da vossa Igreja; dignai-Vos, como é desejo vosso, dar-lhe a paz e unidade, Vós que, sendo Deus, viveis e reinais por todos os séculos dos séculos. Amém.
\end{multicols}
\newpage
\begin{center}
	\textbf{PREPARAÇÃO PARA A COMUNHÃO}
\end{center}
\begin{multicols}{2}
	\noindent Dómine Jesu Christe, Fili Dei vivi, qui ex voluntáte Patris, cooperánte Spíritu Sancto, per mortem tuam mundum vivificásti: líbera me per hoc sacrosánctum Corpus et Sánguinem tuum ab ómnibus iniquitátibus meis, et univérsis malis: et fac me tuis semper inhærére mandátis, et a te numquam separári permíttas: Qui cum eódem Deo Patre et Spíritu Sancto vivis et regnas Deus in saécula sæculórum. Amen.
	\\
	\\ Percéptio Córporis tui, Dómine Jesu Christe, quod ego indígnus súmere præsúmo, non mihi provéniat in judícium et condemnatiónem: sed pro tua pietáte prosit mihi ad tutaméntum mentis et córporis, et ad medélam percipiéndam: Qui vivis et regnas cum Deo Patre in unitáte Spíritus Sancti Deus, per ómnia saécula sæculórum. Amen.
	\\
	\\Senhor Jesus Cristo, Filho do Deus vivo, que, por vontade do Pai e com a cooperação do Espírito Santo, destes com a vossa morte a vida ao mundo, livrai-me, por este vosso sacrossanto Corpo e Sangue, de todas as minhas iniquidades e de todos os males; fazei que eu seja sempre fiel cumpridor dos vossos mandamentos e não permitais que jamais me afaste de Vós, que, com o mesmo Deus Pai e o Espírito Santo, viveis e reinais, Deus, pelos séculos dos séculos. Amém.
	\\
	\\ Que a comunhão do vosso Corpo e Sangue, Senhor Jesus Cristo, que eu, embora indigno, ouso receber, não seja para juízo e condenação minha, mas antes, pela vossa misericórdia, me sirva de proteção e remédio para a alma e para o corpo, Vós que, sendo Deus, viveis e reinais com Deus Pai na unidade do Espírito Santo, por todos os séculos dos séculos. Amém.
\end{multicols}

\begin{center}
	\textbf{COMUNHÃO DO CELEBRANTE}
\end{center}
\begin{multicols}{2}
	\noindent Panem cæléstem accípiam, et nomen Dómini invocábo.
	\\ \textit{(Bate três vezes no peito, dizendo:)}
	\\ Dómine, non sum dignus, ut intres sub tectum meum: sed tantum dic verbo, et sanábitur ánima mea. (3x)  
	\\Tomarei o pão do céu, e invocarei o nome do Senhor.
	\\ Senhor, eu não sou digno de que entreis em minha morada; mas dizei uma só palavra, e minha alma será salva. (3x)
\end{multicols}
\begin{flushleft}
	\textit{Faz sobre si o sinal da cruz com a sagrada Hóstia, antes de a comungar:}
\end{flushleft}
\begin{multicols}{2}
	\noindent Corpus Dómini nostri Jesu Christi custódiat $ \dag$ ánimam meam in vitam ætérnam. Amen. 
	\\O Corpo de nosso Senhor Jesus Cristo guarde $ \dag$ a minha alma para a vida eterna. Amém. 
\end{multicols}
\begin{flushleft}
	\textit{Depois de receber o Corpo de Nosso Senhor, toma o Cálice e diz:}
\end{flushleft}
\begin{multicols}{2}
	\noindent Quid retríbuam Dómino pro ómnibus quæ retríbuit mihi? Cálicem salutáris accípiam, et nomen Dómini invocábo. Laudans invocábo Dóminum, et ab inimícis meis salvus ero. 
	\\ Que hei-de eu retribuir ao Senhor por tudo quanto Ele me concedeu? Tomarei o cálix da salvação, e invocarei o nome do Senhor. Em louvores invocarei o Senhor, e livre serei dos meus inimigos.	
\end{multicols}
\begin{flushleft}
	\textit{
		Faz o sinal da Cruz com o Cálice e diz:}
\end{flushleft}
\begin{multicols}{2}
	\noindent Sanguis Dómini nostri Jesu Christi $ \dag$ custódiat ánimam meam in vitam ætérnam. Amen. 
	\\ O Sangue de Nosso Senhor Jesus Cristo $ \dag$ guarde a minha alma para a vida eterna. Amém.	
\end{multicols}
\begin{center}
	\textbf{COMUNHÃO DOS FIÉIS}
\end{center}
\begin{multicols}{2}
	\noindent \textbf{Confiteor Deo omnipotenti, beatæ Mariæ semper Virgini, beato Michæli Archangelo, beato Joanni Baptistæ, sanctis Apóstolis Petro et Paulo, omnibus Sanctis, et tibi, pater: quia peccavi nimis cogitátione, verbo, et ópere: mea culpa, mea culpa, mea máxima culpa. Ideo precor beatam Mariam semper Virginem, beatum Michælem Archangelum, beatum Joannem Baptistam, sanctos Apóstolos Petrum et Paulum, omnes Sanctos, et te, pater, orare pro me ad Dóminum Deum nostrum.}
	\\ EU, PECADOR, me confesso a Deus todo-poderoso, à bem-aventurada sempre Virgem Maria, ao bemaventurado são Miguel Arcanjo, ao bem-aventurado são João Batista, aos santos apóstolos são Pedro e são Paulo, a todos os Santos, e a vós padre, que pequei muitas vezes, por pensamentos, palavras, obras e omissões, por minha culpa, minha culpa, minha máxima culpa. \\ Portanto, peço e rogo à bem-aventurada sempre Virgem Maria, ao bem-aventurado são Miguel Arcanjo, ao bem-aventurado são João Batista, aos santos apóstolos são Pedro e são Paulo, a todos os Santos, e a vós padre, que rogueis por mim a Deus Nosso Senhor. 
\end{multicols}
\begin{multicols}{2}
	\noindent Misereátur vestri omnípotens Deus, et dimissis peccáis vestris, perdúcat vos ad vitam ætérnam. \textbf{R. Amen.}
	\\
	\\Indulgéntiam, $\dag$ absolutiónem, et
	remissiónem peccatórum nostrorum,
	tríbuat nobis omnípotens et miséricors
	Dominus: \textbf{R. Amen.}
	\\
	\\Deus todo poderoso tenha compaixão de vós, perdoe os vossos pecados, e vos conduza à vida eterna.
	\\ R. Amém.
	\\
	\\Indulgência, $\dag$ absolvição, e remissão dos nossos pecados, conceda-nos o Senhor onipotente e misericordioso. R. Amém.
\end{multicols}
\begin{flushleft}
	\textit{
		O celebrante volta-se para o altar, genuflecte e voltando-se pra os assistentes ergue a Hóstia,
		dizendo:}
\end{flushleft}

\begin{multicols}{2}
	\noindent Ecce Agnus Dei, ecce qui tollit peccáta mundi.
	\\
	\\ \textbf{	Dómine, non sum dignus, ut intres sub tectum meum: sed tantum dic verbo, et sanábitur ánima mea. (3x)}
	\\ Eis o Cordeiro de Deus, eis Aquele que tira os pecados do mundo.
	\\
	\\ Senhor, eu não sou digno de que entreis em minha morada, mas dizei uma só palavra e a minha alma será salva. (3x)
\end{multicols}
\begin{flushleft}
	\textit{O sacerdote diz a cada um dos comungantes:}
\end{flushleft}

\begin{multicols}{2}
	\noindent Corpus Dómini nostri Jesu Christi $\dag$ custódiat ánimam tuam in vitam ætérnam. Amen.
	\\O Corpo de nosso Senhor Jesus Cristo guarde a tua alma para a vida eterna. Amém.
\end{multicols}
\begin{center}
	\textbf{ABLUÇÕES}
\end{center}
\begin{flushleft}
	\textit{Enquanto purifica o cálice o sacerdote reza:}
\end{flushleft}
\begin{multicols}{2}
	\noindent Quod ore súmpsimus, Dómine, pura mente capiámus: et de múnere temporáli fiat nobis remédium sempitérnum. 
	\\ Corpus tuum, Dómine, quod sumpsi, et Sanguis, quem potávi, adhaéreat viscéribus meis: et praésta; ut in me non remáneat scélerum mácula,  
	\\ Com pureza de alma recebamos, Senhor, o que em nossa boca tomamos. Este dom, que nos foi concedido no tempo, remédio nos seja para a eternidade.
	\\ O vosso Corpo, Senhor, que eu comi e o vosso Sangue que eu bebi se unam às minhas entranhas; refeito
	\\quem pura et sancta refecérunt sacraménta: Qui vivis et regnas in saécula sæculórum. Amen.
	\\
	\\
	\\ que fui com estes puros e santos sacramentos, fazei que em mim não fique mancha alguma de pecado, Vós que viveis e reinais pelos séculos dos séculos. Amém.
\end{multicols}
\begin{flushleft}
	\textit{Depois de mudado o missal, o sacerdote reza as orações de ação de graças, chamadas “Communio” e “Post-Communio”. \textbf{Podem os fiéis sentar-se.}}
\end{flushleft}
\begin{center}
	\textbf{COMUNHÃO}
	\\Ps. 127, 4 et 6.
\end{center}
\begin{multicols}{2}
	\noindent Ecce, sic benedicétur omnis homo, qui timet Dóminum: et vídeas fílios filiórum tuórum: pax super Israel (T.P. Allelúja).
	\\ Eis como será abençoado todo o que teme o Senhor: possas tu ver os filhos de teus filhos e a paz em Israel (T.P. Aleluia).
\end{multicols}
\begin{center}
	\textbf{PÓS-COMUNHÃO}
\end{center}
\begin{multicols}{2}
	\noindent Dóminus vobíscum.
	\\\textbf{R. Et cum spíritu tuo. }
	\\ Quáesumus, omnípotens Deus: institúta providéntiae tuae pio favóre comitáre; ut, quos legítima societáte connéctis, longaéva pace custódias. Per Dóminum nostrum Jesum Christum p\textit{er ómnia saécula saeculorum.}
	\\ \textbf{R. Amen }
	\\
	\\
	\\
	\\O Senhor seja convosco.
	\\R. E com o teu espírito.
	\\ Nós Vos pedimos, Deus onipotente, façais seguir de Vossa graça os institutos de Vossa providência, e aos unidos em legítimo consórcio, conservai-os em duradoura paz. Por Nosso Senhor Jesus Cristo, vosso Filho, que, sendo Deus, convosco vive e reina na unidade do Espírito Santopor todos os séculos dos séculos.
	\\ R. Amém.
\end{multicols}