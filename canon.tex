\begin{center}
	\textbf{CÂNON}
\end{center}
\begin{flushleft}
\textit{O Cânon constitui a parte central da Missa. Com o Prefácio, começa a grande <prece eucarística>, a solene oração sacerdotal da Igreja e oblação propriamente dita do Sacrifício.}
\end{flushleft}

\begin{multicols}{2}
	\noindent Dóminus vobíscum.
	\\ \textbf{R. Et cum spíritu tuo.}
	\\ Sursum corda.
	\\ \textbf{R. Habémus ad Dóminum.}
	\\ Grátias agámus Dómino Deo nostro
	\\ \textbf{R. Dignum et justum est.}
	\\O Senhor seja convosco.
	\\R. E com o teu espírito
	\\ Corações ao alto.
	\\ R. Temo-los para o Senhor 
	\\ Demos graças ao Senhor, nosso Deus.
	\\ R. É digno e justo.
\end{multicols}

\begin{center}
	\textit{PREFÁCIO COMUM}
\end{center}

\begin{multicols}{2}
	\noindent Vere dignum et justum est, aequum et salutáre, nos tibi semper, et ubíque grátias ágere: Dómine sancte, Pater omnípotens,  aetérne Deus: per Christum Dóminum nostrum. Per quem majestátem tuam laudant Angeli, adórant Dominatiónes, tremunt Potestátes. Caeli, caelorúmque Virtútes, ac beáta Séraphim, sócia exsultatióne concélebrant. Cum quibus et nostras vocês, ut admítti júbeas, deprecámur, súpplici confessióne dicéntes:
	\\
	\\ É verdadeiramente digno e justo, é dever e salvação nossa, Vos demos graças sempre e em toda parte, Senhor, Pai santo, Deus onipotente e eterno, por Jesus Cristo nosso Senhor, por quem louvam os Anjos a vossa majestade, a adoram as Dominações, a reverenciam as Potestades, a celebram os Céus e as Forças celestes, com os bem-aventurados Serafins, unidos todos em comum exultação. Juntas com as deles, Vos pedimos aceiteis as nossas vozes, que em súplice louvor Vos aclamam:
\end{multicols}

\begin{flushleft}
	\textbf{\textit{De joelhos.}}
\end{flushleft}

\begin{multicols}{2}
	\noindent Sanctus, Sanctus, Sanctus, Dóminus Deus Sábaoth. Pleni sunt cæli et terra glória tua. Hosánna in excélsis. 
	Benedíctus, qui venit in nómine Domini. Hosánna in excélsis.
	\\ Santo, santo, santo é o Senhor Deus das milícias celestes. Cheios estão céu e terra da vossa glória. Hosana nas alturas! Bendito o que vem em o nome do Senhor! Hosana nas alturas!
\end{multicols}

\begin{center}
	\textit{\textbf{CÃNON – ANTES DA CONSAGRAÇÃO}}
\end{center}
\begin{multicols}{2}
\noindent Te ígitur, clementíssime Pater, per Jesum Christum Fílium tuum, Dóminum nostrum, súpplices rogámus ac pétimus, uti accépta hábeas et benedícas, hæc + dona, hæc + múnera, hæc sancta + sacrifícia illibáta. 
\\
\\In primis, quae tibi offérimus pro Ecclésia tua sancta cathólica: quam pacificáre, custodíre, adunáre et régere dignéris toto orbe terrárum: una cum fámulo tuo Papa nostro N. et Antístite nostro N. et ómnibus orthodóxis, atque cathólicae et apostólicae fídei cultóribus.
\\
\\A vós, Pai clementíssimo, por Jesus Cristo vosso Filho e Senhor nosso, humildemente rogamos e pedimos aceiteis e abençoeis estes + dons, estas + dádivas, estas santas + oferendas ilibadas. 
\\
\\Nós Vo-los oferecemos, em primeiro lugar, pela vossa santa Igreja católica, à qual vos dignai conceder a paz, proteger, conservar na unidade e governar, através do mundo inteiro, e também pelo vosso servo o nosso Papa N., pelo nosso Bispo N., e por todos os (bispos) ortodoxos, aos quais incumbe a guarda da fé católica e apostólica.
\end{multicols}
\begin{flushleft}
	\textbf{\textit{Preces pelas pessoas que se recomendaram às nossas orações.}}
\end{flushleft}
\begin{multicols}{2}
	\noindent Meménto, Dómine, famulórum famularúmque tuárum N. et N. et ómnium circumstántium, quorum tibi fides cógnita est, et nota devótio, pro quibus tibi offérimus: vel qui tibi ófferunt hoc sacrifícium laudis, pro se, suísque ómnibus: pro redemptióne animárum suárum, pro spe salútis, et incolumitátis suæ: tibíque reddunt vota sua ætérno Deo, vivo et vero.
	\\
	\\Lembrai-vos, Senhor, dos vossos servos e servas N. e N., e de todos aqueles que estão aqui presentes, cuja fé Vos é conhecida e manifesta a devotação. Por eles Vos oferecemos, ou eles próprios Vos oferecem, este sacrifício de louvor, por si e por todos os seus, para redenção das suas almas, para terem a salvação e incolumidade que esperam; para isso, a Vós dirigem as suas preces, Deus eterno, vivo e verdadeiro.
\end{multicols}
\begin{flushleft}
	\textbf{\textit{Invocando os santos dos céus:}}
\end{flushleft}
\begin{multicols}{2}
	\noindent Communicántes, et memóriam venerántes, in primis gloriósæ semper Vírginis Maríæ, Genitrícis Dei et Dómini nostri Jesu Christi: sed et beáti Joseph, ejúsdem Vírginis Sponsi, et beatórum Apostolórum ac Mártyrum tuórum, Petri et Pauli, Andréæ, Jacóbi, Joánnis, Thomæ, Jacóbi, Philíppi, Bartholomaéi, Mattáei, Simónis, et Thaddaéi, Lini, Cleti, Cleméntis, Xysti, Cornélii, Cypriáni, Lauréntii, Chrysógoni, Joánnis et Pauli, Cosmæ et Damiáni, et ómnium Sanctórum tuórum; quorum méritis precibúsque concédas, ut in ómnibus protectiónis tuæ muniámur auxílio. Per eúmdem Christum Dóminum nostrum. Amen. 
	\\Unidos na mesma comunhão, honramos a memória, em primeiro lugar, da gloriosa sempre Virgem Maria, Mãe de Deus e Senhor Nosso Jesus Cristo, e também de S. José, o Esposo da mesma Virgem, e dos vossos bem-aventurados Apóstolos e Mártires, Pedro e Paulo, André, Tiago, João, Tomé, Tiago, Filipe, Bartolomeu, Mateus, Simão e Tadeu, Lino, Cleto, Clemente, Sixto, Cornélio, Cipriano, Lourenço, Crisógono, João e Paulo, Cosme e Damião, e de todos os vossos Santos. Por seus méritos e preces, concedei sejamos sempre fortalecidos com o vosso auxílio e proteção. Por Jesus Cristo, Senhor nosso. Amém. 
\end{multicols}
\begin{multicols}{2}
	\noindent Hanc ígitur oblatiónem servitútis nostræ, sed et cunctæ famíliæ tuæ, quaésumus, Dómine, ut placátus accípias: diésque nostros in tua pace dispónas, atque ab ætérna damnatióne nos éripi, et in electórum tuórum júbeas grege numerári. Per Christum Dóminum nostrum. Amen.
	\\Quam oblatiónem tu, Deus, in ómnibus, quaésumus, bene+díctam, adscrí$\dagger$ptam, ra$\dagger$tam, rationábilem, acceptabilémque fácere dignéris: ut nobis Cor$\dagger$pus, et San$\dagger$guis fiat dilectíssimi Fílii tui Dómini nostri Jesu Christi. 
	\\Esta oblação, que nós, vossos servos, e toda a vossa família, Vos oferecemos, aceitai-a, Senhor, benignamente; firmai na paz os dias da nossa vida, livrai-nos da eterna condenação e ordenais sejamos contados na sociedade dos vossos eleitos. Amém.
	\\Dignai-Vos, Senhor, fazer que esta oblação seja em tudo aben$\dagger$çoada, apro$\dagger$vada, ratifi$\dagger$cada, espiritual e digna da vossa aceitação, e se torne para nós Cor$\dagger$po e San$\dagger$gue do vosso diletíssimo Filho e Senhor nosso Jesus Cristo.
\end{multicols}
\newpage
\begin{center}
	\textbf{CONSAGRAÇÃO}
\end{center}
\begin{flushleft}
	\textbf{\textit{É o momento mais solene da Missa. O pão e o vinho vão mudar-se substancialmente no Corpo e no Sangue de Jesus. Assim se renova de modo místico a imolação que outrora se realizou no Calvário.}}
\end{flushleft}
\begin{multicols}{2}
	\noindent Qui prídie quam paterétur, accépit panem in sanctas ac venerábiles manus suas, et elevátis óculis in cælum ad te Deum Patrem suum omnipoténtem, tibi grátias agens, bene$\dag$díxit, fregit, dedítque discípulis suis, dicens: Accípite, et manducáte ex hoc omnes.	
	\begin{center}
		\textbf{HOC EST ENIM CORPUS MEUM}
	\end{center}
	Ele, na véspera de sua paixão, tomou o pão em suas santas e veneráveis mãos, e elevando os olhos ao céu para vós, ó Deus, seu Pai onipotente, dando-vos graças, ben$\dag$zeu-o, partiu-o e deu-o a seus discípulos, dizendo: Tomai e Comei Dele, todos.
	\begin{center}
		ISTO É O MEU CORPO
	\end{center} 
\end{multicols}
\begin{flushleft}
	\textbf{\textit{Consagração do Cálice:.}}
\end{flushleft}
\begin{multicols}{2}
	\noindent Símili modo postquam cænátum est, accípiens et hunc præclárum cálicem in sanctas ac venerábiles manus suas: item tibi grátias agens, bene $\dag$ díxit, dedítque discipulis suis, dicens: Accípite, et bíbite ex eo omnes
	\begin{center}
		\textbf{HIC EST ENIM CALIX SANGUINIS
			MEI, NOVI ET ÆTÉRNI
			TESTAMÉNTI: MYSTÉRIUM FIDEI:
			QUI PRO VOBIS ET PRO MULTIS
			EFFUNDÉTUR IN REMISSIÓNEM
			PECCATÓRUM.}
	\end{center}
	Hæc quotiescumque fecérit, in mei
	memóriam faciétis.
	\\
	\\ De igual modo, depois de haver ceado, tomando também este precioso cálice em suas santas e veneráveis mãos, e novamente dando-vos graças, ben $\dag$ zeu-o e deu-o a seus discípulos, dizendo: Tomai e Bebei Dele Todos.
	\begin{center}
		ESTE É O CÁLICE DO MEU SANGUE,
		DO SANGUE DA NOVA E ETERNA
		ALIANÇA:(MISTÉRIO DA FÉ!) O
		QUAL SERÁ DERRAMADO POR VÓS
		E POR MUITOS, PARA A REMISSÃO
		DOS PECADOS.
	\end{center} 
	Todas as vezes que isto fizerdes, fazei-o
	em memória de mim.
\end{multicols}
\begin{flushleft}
	\textbf{\textit{O celebrante continua depois as orações do Cânon:}}
\end{flushleft}
\begin{multicols}{2}
	\noindent Unde et mémores, Dómine, nos servi tui sed et plebs tua sancta, ejúsdem Christi Fílii tui Dómini nostri tam beátæ passiónis, nec non et ab ínferis resurrectiónis, sed et in cælos gloriósæ ascensiónis: offérimus præcláræ majestáti tuæ de tuis donis ac datis, hóstiam $\dag$ puram, hóstiam $\dag$ sanctam, hóstiam $\dag$ immaculátam, Panem $\dag$ sanctum vitæ ætérnæ, et $\dag$ Cálicem salútis perpétuæ. 
	\\
	\\ Supra quæ propítio ac seréno vultu respícere dignéris: et accépta habére, sícuti accépta habére dignátus es múnera púeri tui justi Abel, et sacrifícium patriárchæ nostri Abrahæ: et quod tibi óbtulit summus sacérdos tuus Melchísedech, sanctum sacrifícium, immaculátam hóstiam.
	\\
	\\ Por este motivo, Senhor, nós, vossos servos, e o vosso povo santo, recordando a feliz Paixão do mesmo Jesus Cristo, vosso Filho e Senhor nosso, bem como a sua Ressurreição de entre os mortos e a sua gloriosa Ascenção aos céus, oferecemos à vossa preclara Majestade, dos dons de que Vós próprio nos fizestes mercê, a hóstia $\dag$ pura, hóstia $\dag$ santa, hóstia $\dag$ imaculada, o pão $\dag$ santo da vida eterna e o cálix da eterna $\dag$ salvação.
	\\
	\\ Sobre estas ofertas, dignai-Vos lançar olhar propício e complacente; aceitai-as, assim como Vos dignastes aceitar os dons do justo Abel, vosso servo, o sacrifício de Abraão, nosso pai, e o que vos ofereceu o vosso sumo sacerdote Melquisedeque, sacrifício santo, hóstia imaculada.
\end{multicols}
\begin{flushleft}
	\textit{Profundamente inclinado, o celebrante diz:}
\end{flushleft}
\begin{multicols}{2}
	\noindent Súpplices te rogámus, omnípotens Deus: jube hæc perférri per manus sancti Angeli tui in sublíme altáre tuum, in conspéctu divínæ majestátis tuæ: ut quoquot ex hac altáris participatióne sacrosánctum Fílii tui Cor$\dag$pus, et Sán$\dag$guinem sumpsérimus, omni benedictióne cælésti et grátia repleámur. Per eúmdem Christum Dóminum nostrum. Amen.
	\\
	\\ Suplicantes Vos rogamos, Deus onipotente, façais que estas ofertas sejam levadas pelas mãos do vosso santo Anjo para o vosso sublime altar, à presença da vossa divina Majestade, a fim de que todos nós, que, comungando deste altar, recebermos o sacrossanto Cor$\dag$po e San$\dag$gue do vosso Filho, sejamos cumulados de todas as bênçãos e graças celestes. Por Jesus Cristo Senhor nosso. Amém.
\end{multicols}
\begin{flushleft}
	\textbf{\textit{Memento dos defuntos:}}
\end{flushleft}
\begin{multicols}{2}
	\noindent Meménto étiam, Dómine, famulórum famularúmque tuárum N. et N. qui nos præcessérunt cum signo fídei, et dórmiunt in somno pacis.
	\\ Ipsis, Dómine, et ómnibus in Christo
	quiescéntibus, locum refrigérii, lucis et pacis,
	ut indúlgeas , deprecámur. Per eúmdem
	Christum Dóminum nostrum. Amen.
	\\
	\\
	\\ Lembrai-Vos também, Senhor, dos vossos servos e servas (NN. e NN.), que nos precederam, marcados com o sinal da fé, e agora dormem no sono da paz. 
	\\ A estes, Senhor, e a todos aqueles que repousam em Cristo, Vos pedimos concedais misericordioso o lugar do refrigério, da luz e da paz. Pelo mesmo Jesus Cristo Senhor nosso. Amém.
\end{multicols}
\begin{flushleft}
	\textit{O celebrante bate no peito, dizendo:}
\end{flushleft}
\begin{multicols}{2}
	\noindent Nobis quoque peccatóribus fámulis tuis, de multitúdine miseratiónum tuárum sperántibus, partem áliquam, et societátem donáre dignéris, cum tuis sanctis Apóstolis et Martýribus: cum Joánne, Stéphano, Matthía, Bárnaba, Ignátio, Alexándro, Marcellíno, Petro, Felicitáte, Perpétua, Agatha, Lúcia, Agnéte, Cæcília, Anastásia, et ómnibus Sanctis tuis: intra quorum nos consórtium, non æstimátor mériti, sed véniæ, quaésumus, largítor admítte. Per Christum Dóminum nostrum.
	\\ A nós também, pecadores, vossos servos, confiados nas vossas infinitas misericórdias, dignai-Vos conceder entremos a fazer parte da sociedade dos vossos Santos Apóstolos e Mártires, João, Estêvão, Matias, Barnabé, Inácio, Alexandre, Marcelino, Pedro, Felicidade, Perpétua, Águeda, Lúcia, Inês, Cecília, Anastásia e todos os vossos Santos, em cujo consórcio Vos pedimos nos admitais com vossa liberalidade, não já em consideração dos nossos méritos, mas sim pela vossa indulgência.Por Jesus Cristo Senhor nosso.
\end{multicols}
\newpage
\begin{flushleft}
	\textbf{\textit{DOXOLOGIA FINAL}}
\end{flushleft}
\begin{multicols}{2}
	\noindent Per quem hæc ómnia Dómine, semper bona creas, sanctí$\dag$ficas, viví$\dag$ficas, bene$\dag$dícis, et præstas nobis.
	\\ Per $\dag$ ipsum, et cum $\dag$ ipso, et in $\dag$ ipso, est tibi Deo Patri $\dag$ omnipoténti, in unitáte $\dag$ Spíritus Sancti, omnis honor et glória. 
	\\ Por Ele, Senhor, criais sempre todos estes bens, os santi$\dag$ficais, vivi$\dag$ficais, aben$\dag$çoais e no-los concedeis.
	\\ Por $\dag$ Ele, com $\dag$ Ele e $\dag$ n’Ele, a Vós, Deus Pai $\dag$ onipotente, na unidade do $\dag$ Espírito Santo, é dada toda a honra e glória.
\end{multicols}