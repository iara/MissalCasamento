\begin{center}
	MISSA DE CASAMENTO
	\\PREPARAÇÃO
\end{center}


\begin{multicols}{2}
\noindent
In nomine Patris, $\dag$  et Fílii, et Spíritus Sancti. Amen.
\\ \textbf{Ant.} Introíbo ad altare Dei.
\\ \textbf{R. Ad Deum qui lætíficat juventútem meam.}
\\ Em nome do $\dag$ Pai, e do Filho, e do Espírito Santo. Amém.
\\ \textbf{Ant.} Vou-me aproximar do altar de Deus.
\\ R. Ao Deus que é a minha alegria.
\end{multicols}
\begin{center}
	\textbf{Salmo 42}
\end{center}
\begin{multicols}{2}
\noindent Júdica me, Deus, et discérne causam meam de gente non sancta: ab hómine iniquo et dolóso érue me. 
\\ \textbf{Quia tu es, Deus, fortitúdo mea: quare me repulísti, et quare tristis incédo, dum afflígit me inimícus?}
\\
\\Emitte lucem tuam et veritátem tuam: ipsa me deduxérunt et adduxérunt in montem sanctum tuum, et in tabernácula tua. 
\\ \textbf{Et introíbo ad altare Dei: ad Deum qui lætíficat juventútem meam.}
\\
\\ Confitébor tibi in cíthara Deus, Deus meus: quare tristis es ánima mea, et quare contúrbas me? 
\\ \textbf{Spera in Deo, quóniam adhuc confitébor illi: salutáre vultus mei, et Deus meus}
\\
\\ Glória Patri, et Fílio, et Spíritui Sancto. 
\\ \textbf{R. Sicut erat in princípio, et nunc, et semper: et in sæcula sæculórum. Amen.}
\\
\\ Julgai-me, ó Deus, e separai a minha causa da causa da gente ímpia. Livraime do homem injusto e enganador. 
\\ R. Pois vós, ó meu Deus, sois a minha força. Por que me repelis? Por que ando eu triste, quando me aflige o inimigo? 
\\
\\ Enviai-me a vossa luz e a vossa verdade. Elas me guiarão e hão de conduzir-me a vossa montanha santa, ao lugar onde habitais. 
\\ R. Entrarei ao altar de Deus, ao Deus que é a minha alegria.
\\ 
\\ Louvar-vos-ei ó Deus, Deus meu, ao som da harpa. Por que estais triste, ó minha alma? E por que me inquietas?
\\ R. Espera em Deus, porque ainda o louvarei como meu Salvador e meu Deus.
\\ Glória ao Pai, ao Filho, e ao Espírito Santo. 
\\ R. Assim como era no princípio, agora e sempre, e por todos os séculos dos séculos. Amém.


\end{multicols}

\noindent \textit{Repete a Antífona:}

\begin{multicols}{2}
\noindent \textbf{Ant.} Introíbo ad altare Dei.
\\ \textbf{R. Ad Deum qui lætíficat juventútem meam.}
\\
\\ Adjutórium $\dagger$ nostrum in nómine Dómini. 
\\ \textbf{R. Qui fecit cælum et terram.}
\\ \textbf{Ant.} Vou-me aproximar do altar de Deus.
\\ R. Ao Deus que é a minha alegria.	
\\
\\ O nosso $\dagger$ auxílio está no nome do Senhor. 
\\ R. Que fez o Céu e a Terra.
\end{multicols}
\textit{Profundamente inclinado, o celebrante diz o Confiteor, e depois dele, os assistentes.
}
\begin{multicols}{2}
\noindent Confíteor Deo omnipotenti, etc.
\\ \textbf{R. Misereátur tui omnípotens Deus, et dimissis peccatis tuis, perducat te ad vitam æternam.}
\\Eu pecador me confesso, etc. 
\\ R. Que Deus onipotente se compadeçade vós, perdoe os vossos pecados e vos conduza à vida eterna. 
\end{multicols}
	\textit{Celebrante: Amen. }
\begin{multicols}{2}
\noindent \textbf{Confiteor Deo omnipotenti,/ beatæ Mariæ semper Virgini, / beato Michæli Archangelo, / beato Joanni Baptistæ, / sanctis Apóstolis Petro et Paulo, / omnibus Sanctis, et tibi, pater: / quia peccavi nimis cogitátione, verbo, et ópere: / mea culpa, mea culpa, mea máxima culpa. 
\\ Ideo precor beatam Mariam semper Virginem, / beatum Michælem Archangelum, / beatum Joannem Baptistam, / sanctos Apóstolos Petrum et Paulum, / omnes Sanctos, et te, pater, / orare pro me ad Dóminum Deum nostrum.}
\\
\\ EU, PECADOR, me confesso a Deus todo-poderoso, à bem-aventurada sempre Virgem Maria, ao bemaventurado são Miguel Arcanjo, ao bem-aventurado são João Batista, aos santos apóstolos são Pedro e são Paulo, a todos os Santos, e a vós padre, que pequei muitas vezes, por pensamentos, palavras, obras e omissões, por minha culpa, minha culpa, minha máxima culpa. \\ Portanto, peço e rogo à bem-aventurada sempre Virgem Maria, ao bem-aventurado são Miguel Arcanjo, ao bem-aventurado são João Batista, aos santos apóstolos são Pedro e são Paulo, a todos os Santos, e a vós padre, que rogueis por mim a Deus Nosso Senhor. 
\end{multicols}

\textit{Celebrante:}

\begin{multicols}{2}
\noindent Misereátur vestri omnípotens Deus, et dimissis peccáis vestris, perdúcat vos ad vitam ætérnam. \textbf{R. Amen.}
\\
\\Indulgéntiam, $\dag$ absolutiónem, et
remissiónem peccatórum nostrorum,
tríbuat nobis omnípotens et miséricors
Dominus: \textbf{R. Amen.}
\\
\\Deus todo poderoso tenha compaixão de vós, perdoe os vossos pecados, e vos conduza à vida eterna.
\\ R. Amém.
\\
\\Indulgência, $\dag$ absolvição, e remissão dos nossos pecados, conceda-nos o Senhor onipotente e misericordioso. R. Amém.
\end{multicols}

\textit{O celebrante, inclinado, diz:}
\begin{multicols}{2}
\noindent Deus, tu convérsus vivificábis nos.
\\ \textbf{R. Et plebs tua lætábitur in te.}
\\
\\ Osténde nobis Dómine, misericordiam tuam. 
\\ \textbf{R. Et salutáre tuum da nobis.}
\\
\\ Dómine, exáudi oratiónem meam.
\\ \textbf{R. Et clámor meus ad te véniat}
\\
\\ Dominus vobiscum.
\\ \textbf{R. Et cum spíritu tuo}
\\
\\
\\ Ó Deus, voltando-vos para nós nos dareis a vida.
\\ R. E o vosso povo se alegrará em vós. Deus, tu convérsus vivificábis nos.
\\
\\ Mostrai-nos, Senhor, a vossa misericórdia.
\\ R. E dai-nos a vossa salvação.
\\
\\ Ouvi, Senhor, a minha oração.
\\ R. E chegue até vós o meu clamor.
\\
\\ O Senhor seja convosco.
\\ R. E com o vosso espírito
\end{multicols}
\textit{O celebrante sobe ao altar, dizendo:}
\begin{multicols}{2}
\noindent
Oremus.
\\Aufer a nobis, quæsumus, Dómine, iniquitates nostras: ut ad Sancta sanctórum puris mereámur méntibus intróire. Per Christum Dóminum nostrum. Amen. 
\\Oremos.
\\Pedimos-vos, Senhor, afasteis de nós as nossas iniqüidades, para que, com
almas puras, mereçamos entrar no Santo dos Santos. Por Cristo Jesus Nosso Senhor. Amém 
\end{multicols}
\textit{O celebrante, inclinado, diz a seguinte oração:}
\begin{multicols}{2}
\noindent Orámus te, Dómine, per mérita Sanctórum tuórum, quórum relíquiæ hic	sunt, et ómnium Sanctórum: ut indulgére	dignéris ómnia peccáta mea. Amen. 
\\
\\Nós vos suplicamos, Senhor, pelosméritos de vossos santos, cujas relíquias aqui se encontram, e de todos os demais santos, vos digneis perdoar todos os nossos pecados. Amém.
 
\end{multicols}
