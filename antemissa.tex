\newpage
\begin{center}
	PRIMEIRA PARTE: ANTE-MISSA
	\\INTRÓITO
\end{center}
\begin{center}
\textbf{KYRIE ELEISON}
\end{center}

\begin{multicols}{2}
\noindent
Kyrie eléison.
\\Kyrie eléison.
\\Kyrie eléison.
\\
\\Christe eléison.
\\Christe eléison.
\\Christe eléison.
\\
\\Kyrie eléison.
\\Kyrie eléison.
\\Kyrie eléison.
\\Senhor, tende piedade de nós.
\\Senhor, tende piedade de nós.
\\Senhor, tende piedade de nós.
\\
\\Cristo, tende piedade de nós.
\\Cristo, tende piedade de nós.
\\Cristo, tende piedade de nós.
\\
\\Senhor, tende piedade de nós.
\\Senhor, tende piedade de nós.
\\Senhor, tende piedade de nós.
\end{multicols}
\begin{center}
	\textbf{GLORIA IN EXCELSIS}
\end{center}
\begin{multicols}{2}
\noindent 
GLÓRIA IN EXCÉLSIS DEO, / et in terra pax homínibus / bonæ voluntátis. /
Laudámus te, / benedícimus te, adorámus te, / glorificámus te, / grátias ágimus tibi /
propter magnam glóriam tuam: / Dómine Deus, / Rex cæléstis, / Deus Pater omnípotens. / 
Dómine Fili unigénite, /
Jesu Christe; / Dómine Deus, / Agnus Dei,
/ Fílius Patris: / Qui tollis peccata mundi, /
miserére nobis; / qui tollis peccáta mundi,
/ súscipe deprecatiónem nostram; / qui sedes ad déxteram Patris, / miserére nobis.
/ Quóniam tu solus Sanctus, tu solus
Dóminus, / tu solus Altíssimus, Jesu Christe, cum Sancto Spíritu: $\dag$ / in glória Dei Patris. / Amen.
\\ GLÓRIA A DEUS NAS ALTURAS; e na
terra paz aos homens de boa vontade.
Nós Vos
louvamos,
nós
Vos
bendizemos, nós Vos adoramos, nós Vos
glorificamos, nós Vos damos graças, por
Vossa imensa glória, Senhor Deus, Rei
dos céus, Deus Pai onipotente. Senhor
Filho Unigênito. Senhor Deus, Cordeiro
de Deus, Filho de Deus Pai. Vós que
tirais o pecado do mundo, tende
piedade de nós. Vós que tirais o pecado
do mundo, acolhei a nossa súplica. Vós
que estais à direita do Pai, tende
piedade de nós. Só vós sois Santo. Só
vós sois o Senhor. Só vós o Altíssimo,
Jesus Cristo. Com o Espírito Santo $\dag$ ,
na glória de Deus Pai. Amém.
\end{multicols}

\begin{multicols}{2}
	\noindent
	V. Dóminus vobíscum.
	\\R. Et cum spíritu tuo.
	\\Oremus.
	\\
	\\V. O Senhor seja convosco
	\\R. E com vosso espírito.
	\\Oremos.
\end{multicols}
\begin{center}
	\textbf{COLETA}
	\\\textit{O celebrante, diante do missal, recita a COLETA. Breve oração que resume e apresenta a Deus os votos de
	toda a assembléia, votos estes sugeridos pelo mistério ou solenidade do dia.}
\end{center}
\begin{multicols}{2}
	\noindent
	Exáudi nos, omnípotens et miséricors Deus: ut, quod nostro ministrátur offício, tua benedictióne pótius impleátur. Per Dóminum nostrum \textit{per ómnia sǽculua sæculórum.}
	\\ \textbf{R. Amen}
	\\
	\\
	\\Ouvi-nos, Senhor Deus, misericordioso e onipotente, e dai cumprimento por Vossa bênção ao que operamos por nosso ministério. Por Nosso Senhor Jesus Cristo, vosso Filho, que, sendo Deus, convosco vive e reina na unidade do Espírito Santo, por todos os séculos dos séculos. 
	\\R. Amém.
\end{multicols}
\begin{center}
	\textbf{EPÍSTOLA}
	\\\textit{A Epístola é a leitura das Sagradas Esrituras que se faz ao povo para intruí-lo e prepará-lo melhor para o Sacrifício.}
\end{center}
\newpage
\begin{multicols}{2}
	\noindent
	\textit{Léctio Epístolae B.Pauli Apóstoli ad Ephésios.}
	\\ Fratres : Mulíeres viris suis súbditae sint, sicut Dómino; quóniam vir caput est mulíeris, sicut Christus caput est Ecclésiae: Ipse, salvátor córporis ejus. Sed sicut Ecclésia subjécta est Christo, ita et mulíeres viris suis in ómnibus. Viri, dilígite uxóres vestras, sicut et Christus diléxit Ecclésiam, et seípsum trádidit pro ea, ut illam sanctificáret, mundans lavácro aquae in verbo vitae, ut exhibéret ipse sibi gloriósam Ecclésiam, non habéntem máculam, aut rugam, aut áliquid hujúsmodi, sed ut sit sancta et immaculáta. Ita et viri debent dilígere uxóres suas, ut córpora sua. Qui suam uxórem díligit, seípsum díligit. Nemo enim unquam carnem suam ódio hábuit, sed nutrit, et fovet eam, sicut et Christus Ecclésiam: quia membra sumus córporis ejus, de carne ejus et de óssibus ejus. Propter hoc relínquet homo patrem et matrem suam, et adhaerébit uxóri suae: et erunt duo in carne una. Sacraméntum hoc magnum est, ego autem dico in Christo et in Ecclésia. Verúmtamen et vos sínguli, unusquísque uxórem suam, sicut seípsum díligat: uxor autem tímeat virum suum.	
	\\ \textbf{	R. Deo grátias. }
	\\
	\\
	\\
	\\ \textit{Leitura da Epístola de S.Paulo Apóstolo aos Efésios.}
	\\ Irmãos: As mulheres sejam sujeitas a seus maridos como ao Senhor; porque o marido é a cabeça da mulher, assim como Cristo é a cabeça da Igreja: Ele mesmo, que é o Salvador do seu corpo. Assim como, pois, é a Igreja sujeita a Cristo, assim o sejam em tudo as mulheres a seus maridos. Vós, maridos, amai vossas mulheres, como Cristo amou a Igreja, e por ela se entregou, para a santificar, purificando-a no Batismo da água pela palavra da vida; para a apresentar a Si mesmo como Igreja gloriosa, sem mácula, nem ruga, nem outro algum defeito semelhante, mas santa e imaculada. Assim é que também os maridos devem amar a suas mulheres, como a seu próprio corpo. O que ama a sua mulher ama-se a si mesmo. Porque ninguém aborreceu jamais a sua própria carne; mas cada um a nutre e fomenta, como também Cristo faz à sua Igreja. Porque somos membros do Seu corpo, da Sua carne e dos Seus ossos. Por isso o homem deixará seu pai e mãe e se unirá à mulher, e serão dois em uma mesma carne. Este sacramento é grande, mas eu digo em Cristo e na Sua Igreja. Contudo também vós, cada um de per si, ame a sua mulher como a si mesmo; e a mulher reverencie a seu marido.
	\\ R. Graças a Deus
\end{multicols}
\begin{center}
	\textbf{GRADUAL}
\end{center}
\textit{“Gradual” são orações que se dizem entre a Epístola e o Evangelho. Ordinariamente, é algum salmo que serve de preparação para o Evangelho. Assim chamado porque antigamente era cantado nos “degraus” da estande ou do púlpito. É oração móvel conforme a missa.}

\begin{multicols}{2}
	\noindent
	Uxor tua sicut vitis abúndans in latéribus domus tuae. 
	\\V. Fílii tui sicut novéllae olivárum in circúiti mensae tuae.
	\\Tua mulher há-de ser como vide abundante no interior de tua casa. 
	\\V. Teus filhos ao redor da mesa, serão como rebentos de oliveira.
\end{multicols}
\begin{center}
	\textbf{ALELUIA}
\end{center}
\textit{“Alleluia” é uma palavra hebraica que quer dizer: Deus seja louvado. Na missa, vem acompanhada de um versículo de salmo, próprio para cada missa.}
\begin{multicols}{2}
	\noindent
	Allelúja, allelúja. 
	Mittat vobis Dóminus auxílium de sancto: et de Sion tueátur vos. 
	Allelúja.
	\\Aleluia, aleluia. 
	Mande-vos o Senhor ajuda de Seu santuário; e proteja-vos do alto de Sião. Aleluia.
\end{multicols}
\textit{Vai o sacerdote ao meio do altar e reza a preparação do Evangelho:}
\begin{multicols}{2}
	\noindent
	Munda cor meum, ac lábia mea, omnípotens Deus, qui lábia Isaíae Prophétae cálculo mundásti igníto: ita me tua grata miseratióne dignáre mundáre, ut sanctum Evangélium tuum digne váleam nuntiáre. Per Christum Dóminum nostrum. Amen.
	\\Jube, Domine, benedícere. Dóminus sit in corde meo, et in lábiis meis: ut digne et competénter annúntiem Evangélium suum. Amen. 
	\\
	\\Purificai-me, Deus onipotente, o coração e os lábios, Vós que purificastes os lábios do profeta Isaías com um carvão em brasa; pela vossa misericordiosa bondade, dignai-Vos purificar-me, de modo a tornar-me capaz de proclamar dignamente o vosso santo Evangelho.
	\\Dignai-Vos, Senhor, abençoar-me. Esteja o Senhor no meu coração e nos meus lábios, para digna e competentemente proclamar o seu Evangelho. Ámen.
\end{multicols}
\newpage
\begin{center}
	\textbf{EVANGELHO}
\end{center}
\begin{multicols}{2}
	\noindent
Dóminus vobíscum.
\\\textbf{R. Et cum spíritu tuo.}
\\Sequéntia sancti + Evangélii secúndum Mattháeum.
\\\textbf{R. Glória tibi, Domine.}
\\
\\ In illo témpore: Accessérunt ad Jesum Pharisáei, tentántes eum et dicéntes : Si  licet hómini dimíttere uxórem suam, quacúmque ex causa? Qui respóndens, ait eis : Non legístis, quia qui fecit hóminem ab initio, másculum et féminam fecit eos? Et dixit : Propter hoc dimíttet homo patrem et matrem, et adhaerébit uxóri suae, et erunt duo in carne una. Itaque jam non sunt duo, sed una caro. Quod ergo Deus conjúnxit, homo non séparet.
\\\textbf{R. Laus tibi, Christe. }
\\
\\O Senhor seja convosco.
\\R. E com o teu espírito. 
\\Continuação do S. Evangelho segundo S. Mateus.
\\R. Glória a Vós, Senhor.
\\
\\Naquele tempo: Chegaram-se a Jesus os Fariseus, tentando-O e dizendo: É porventura lícito a um homem repudiar a sua mulher por qualquer causa? Ele, respondendo, lhes disse: Não tendes lido que quem criou o homem desde o princípio, os criou homem e mulher? E disse: Por isso deixará o homem pai e mãe e ajuntar-se-á a sua mulher, e serão dois numa só carne. Assim que já não são dois, mas uma só carne. Não separe, logo, o homem o que Deus uniu.
\\R. Louvor a Vós, ó Cristo
\end{multicols}

\begin{multicols}{2}
	\noindent
	Per evangélica dicta deleántur nostra delícta. 
	\\Por estas palavras do Evangelho, perdoados sejam os nossos pecados.
\end{multicols}
\begin{center}
	\textbf{\textit{Após o Evangelho, o celebrante comenta a palavra divina, fazendo a homilia.}}
\end{center}
\newpage
\begin{center}
	\textbf{CREDO}
\end{center}
\begin{multicols}{2}
	\noindent
	CREDO in unum Deum, Patrem
	omnipoténtem, / factórem cæli et terræ, /
	visibílium ómnium et invisibílium.
	Et in unum Dóminum Jesum Christum, /
	Fílium Dei unigenitum. / Et ex Patre
	natum / ante ómnia sæcula. Deum de Deo,
	/ lumen de Lúmine, / Deum verum de Deo
	vero. / Génitum, non factum, /
	consubstantiálem Patri: / per quem ómnia
	facta sunt. / Qui propter nos hómines / et
	propter nostram salútem / descéndit de
	cælis. /
	\\ ET INCARNATUS EST DE SPIRITU
	SANCTO EX MARIA VIRGINE: ET
	HOMO FACTUS EST.
	\\Crucifíxus étiam pro nobis : / sub Póntio
	Piláto / passus, et sepúltus est. / Et
	resurréxit tértia die, / secundum
	Scriptúras. / Et ascéndit in cælum: / sedet
	ad déxteram Patris. / Et íterum ventúrus
	est cum glória / judicáre vivos et mórtuos:
	/ cujus regni non erit finis. /
	\\Et in Spíritum Sanctum, / Dóminum et
	vivificántem: / qui ex Patre, Filióque
	procédit. / Qui cum Patre, et Fílio simul
	adorátur, / et conglorificátur: / qui locutus
	est per Prophétas.
	\\ Et unam, sanctam, cathólicam / et
	apostólicam Ecclésiam. / Confíteor unum
	baptísma / in remissiónem peccatórum. /
	Et exspécto resurrectiónem mortuórum. /
	Et vitam $\dag$ ventúri sæculi. / Amen.
	\\ CREIO em um só Deus, Pai todo
	poderoso, criador do céu e da terra,de
	todas as coisas visíveis e invisíveis.
	Creio em um só Senhor, Jesus Cristo,
	Filho unigênito de Deus,nascido do Pai,
	antes de todos os séculos; Deus de
	Deus, luz da luz, Deus verdadeiro de
	Deus verdadeiro; Gerado, não criado,
	consubstancial ao Pai, por Ele todas as
	coisas foram feitas. Por nós homens, e
	para nossa salvação, desceu dos céus.
	\\(todos se ajoelham) E SE ENCARNOU, ET INCARNATUS EST DE SPIRITU
	SANCTO EX MARIA VIRGINE: ET
	HOMO FACTUS EST.
	PELO ESPÍRITO SANTO, NO SEIO DA
	VIRGEM MARIA, E SE FEZ HOMEM.
	\\Também por amor de nós foi
	crucificado, sob Pôncio Pilatos; padeceu
	e foi sepultado. Ressuscitou ao terceiro
	dia, conforme as Escrituras. E subiu aos
	Céus, onde está sentado à direita do
	Pai. E de novo há de vir, em sua glória,
	para julgar os vivos e os mortos; E o seu
	reino não terá fim.
	\\Creio no Espírito Santo, Senhor que dá
	a vida, e procede do Pai e do Filho; e
	com o Pai e o Filho é igualmente
	adorado e glorificado: ele o que falou
	pelos profetas.
	\\Creio na Igreja, una, santa, católica e
	apostólica. Professo um só Batismo,
	para a remissão dos pecados. E espero a
	ressurreição dos mortos e a vida $\dag$ do
	mundo que há de vir. Amém
\end{multicols}