\begin{center}
	SACRAMENTO DO MATRIMÔNIO
\end{center}

\textit{Nesta benção e na missa de casamento, a liturgia fala das grandezas e das obrigações do matrimônio cristão. S. Paulo compara-o à união de Cristo com a Igreja, sofrendo e morrendo por ela. A benção nupicial exprime, com magnifífca oração, os mais belos votos que exprimir se podem aos esposos cristãos.
\\O padre de sobrepeliz e estola branca, ou, se tiver de celebrar a missa, com vestes sacerdotais sobe ao altar, onde. de pé, estando e voltando para o altar todos os outros de joelho, diz:}

\begin{multicols}{2}
\noindent
Adjutorium nostrum in nomine Domini.
\\\textbf{R. Qui fecit caelum et terram.}
\\
\\Domine, exaudi orationem meam.
\\\textbf{R. et clamor meus ad te veniat.}
\\
\\Dominus vobiscum.
\\\textbf{R. Et cum spiritu tuo.}
\\Toda a nossa esperança está no Nome do Senhor.
\\R. Que fez o Céu e a terra. 
\\
\\Atendei, Senhor, a minha oracao.
\\R. Chegue ate Vos o meu clamor.
\\
\\O Senhor esteja convosco
\\R. E com o teu espírito
\end{multicols}
\noindent
Oremos.
\\Em todas as nossa ações, Senhor, previni-nos com a vossa inspiração e acompanhai-nos com o vosso auxilio, de modo que todas as nossas orações e todas as ações em Vós tenham princípio e por Vós tenham princípio e por Vós cheguem ao termo. Por Cristo Senhor nosso. \textbf{R. Amém.}
\begin{center}
	\textit{Neste momento dirige aos esposos a exortação do costume a seguir à qual interoga-os a respeito do consentimento; primeiro o esposo:}
\end{center}
N., pensou bem, diante de Deus, no ato que vai realizar. Diga-me pois: É de livre vontade que vai contrair matrimônio com sua futura esposa?
\\ \textit{Esposo responde:} \textbf{É sim}
\vspace{3mm}
\\Está deveras decidido a amar e respeitar esta sua futura esposa e guardar-lhe inviolável fidelidade até à sua morte?
\\ \textit{Esposo:} \textbf{Estou sim sim}
\vspace{3mm}
\\Está disposto a receber da mão de Deus os filhos que Ele houver por bem conceder e a educá-los cristãnamente na santa religião católica?
\\ \textit{Esposo responde:} \textbf{Estou sim}
\\ \begin{center}
	\textit{A seguir interorga a Esposa:}
\end{center}
N., pensou bem, diante de Deus, no ato que vai realizar. Diga-me pois: É de livre vontade que vai contrair matrimônio com sua futura esposo?
\\ \textit{Esposa responde:} \textbf{É sim}
\vspace{3mm}
\\Está deveras decidido a amar e respeitar esta seu futura esposo e guardar-lhe inviolável fidelidade até à sua morte?
\\ \textit{Esposa:} \textbf{Estou sim sim}
\vspace{3mm}
\\Está disposto a receber da mão de Deus os filhos que Ele houver por bem conceder e a educá-los cristãnamente na santa religião católica?
\\ \textit{Esposa responde:} \textbf{Estou sim}

\begin{center}
	\textit{Os esposos unem as mãos direitas, que sacerdote cobre com a estola, dirigindo-se em seguida ao esposa, e depois à esposa:}
\end{center}

\noindent
 N., quer receber N., aqui presente, por sua legítima esposa, em conformidade com as leis da Santa Madre Igreja?
\\ \textbf{R. Quero}
\vspace{3mm}
\\N., quer receber N., aqui presente, por seu legítima esposo, em conformidade com as leis da Santa Madre Igreja?
\\ \textbf{R. Quero}
\vspace{3mm}
\begin{center}
	\textit{O padre confirma o compromisso de qye acaba de ser testemunha dizendo:}
\end{center}


\begin{multicols}{2}
\noindent
Ego auctoritáte Ecclésiæ matrimónium per vos contráctum confírmo et benedíco: In nímine Patris, et Fílii, $\dagger$ et Spíritus Sancti. 
\\ \textbf{R. Amen.}
\\
\\Eu, por autoridade da Igreja, confirmo o matrimônio que entre vós acabais de contrair e o abençoo: Em nome do Pai e do Filho $\dagger$ e do Espírito Santo.
\\ R. Amém.
\end{multicols}

\begin{center}
	\textit{Aqui asperge os esposos com água benta. E continua:}
\end{center}
\noindent
E a vos todos aqui presentes eu tomo por testemunhas desta sagrada união:
``O que Deus uniu, jamais o homem o pode separar!"

\begin{multicols}{2}
\noindent
Adiutórium nostrum $\dagger$ in nómine Dómini.
\\\textbf{R. Qui fecit cælum et terram.}
\\
\noindent Dómine, exáudi oratiónem meam.
\\\textbf{R. Et clamor meus ad te véniat.}
\\
\noindent Dóminus vobíscum.
\\\textbf{R. Et cum spíritu tuo.}
\\
\noindent Orémus.
\\Béne$\dagger$dic, Dómine, ánulum hunc, quem nos in tuo nómine bene$\dagger$dícimus: ut, quæ eum gestáverit, fidelitátem íntegram suo sponso tenens, in pace et voluntáte tua permáneat, atque in mútua caritáte semper vivat. Per Christum Dóminum nostrum.
\textbf{R. Amen}

Nosso auxílio está $\dagger$ no nome do Senhor.
R. Que fez o céu e a terra.

Senhor, ouvi minha oração.
R. E meu clamor chegue a Vós.

O Senhor esteja convosco.
R. E com o teu espírito.

Oremos.
Aben$\dagger$çoai, Senhor, estes anéis que
em vosso nome aben$\dagger$çoamos, para
que os esposos que as usarem,
guardem íntegra fidelidade um ao
outro, permaneçam sempre na vossa
paz e submissos à vossa vontade e
vivam sempre em mútua caridade. Por
Cristo Senhor nosso.
R. Amém.
\end{multicols}

\textit{Então o celebrante asperge os anéis com água benta e diz para os esposos:}
\\
\\Unidos para sempre pelos vínculos indissolúveis do Matrimônio cristão, ides
agora entregar um ao outro o anel da fidelidade conjugal. Dizei comigo:

\begin{center}
	\textit{O esposo recebe do celebrante o anel e o coloca no dedo anular esquerdo da esposa dizendo com o celebrante:}
\end{center}

\noindent \textbf{Em nome do Pai e do Filho $\dagger$ e do Espírito Santo, N., recebe esta aliança em sinal do teu amor e da tua fidelidade.}

\begin{center}
	\textit{E procede da mesma maneira com a esposa}
\end{center}

Os esposos se ajoelham e o celebrante procede:

\newpage
\begin{multicols}{2}
\noindent Confírma hoc, Deus, quod operátus es in nobis.
\\ \textbf{R. A templo sancto tuo, quod est in Jerúsalem.}
\\
\\Kýrie, éleison.
\\ \textbf{M. Christe, éleison. Kýrie, éleison.}
\\
\\Pater noster (em voz baixa)
\\Et ne nos indúcas in
tentatiónem.
\\ \textbf{R. Sed líbera nos a malo.}
\\
\\ Salvos fac servos tuos.
\\ \textbf{Deus meus, sperántes in te.}
\\
\\ \textbf{Mitte eis, Dómine, auxílium de sancto.}
\\ \textbf{R. Et de Sion tuére eos.}
\\
\\ Esto eis, Dómine, turris fortitúdinis.
\textbf{R. A fácie inimíci.}
\\
\\Dómine, exáudi oratiónem meam.
\\ \textbf{M. Et clamor meus ad te véniat.}
\\
\\Dóminus vobíscum.
\\ \textbf{R. Et cum spíritu tuo.}
\\
\\Orémus.
\\Réspice, quǽsumus, Dómine, super hos fámulos tuos: et institútis tuis, quibus propagatiónem humáni géneris ordinásti, benígnus assíste; ut qui te auctóre jungúntur, te auxiliánte servéntur. Per Christum Dóminum nostrum. \textbf{R. Amen}
\\
\\
\\Confirmai, Deus, isto que em nós operastes.
\\R. Lá do vosso santuário, que esta em Jerusalém.
\\
\\Senhor, misericórdia.
\\R. Cristo, misericórdia. Senhor misericórdia.
\\
\\Pai nosso (em voz baixa até) 
\\E não nos deixeis cair em tentação.
\\R. Mas livrai-nos do mal.
\\
\\Salvai os vossos servos.
\\R. Que em Vós esperam, meu Deus.
\\
\\Enviai-os, Senhor, o vosso auxílio
do santuário.		
\\R. E de Sião defendei-os
\\
\\Sede para eles, Senhor, fortaleza inexpugnável.
\\R. Ante a face do inimigo.
\\
\\Senhor, ouvi minha oração.
\\R. E meu clamor chegue a Vós.
\\
\\O Senhor esteja convosco.
\\R. E com teu espírito.
\\
\\Oremos.
\\Pousai os olhos, pedimos, Senhor, sobre estes vossos servos, e assisti benignamente vossa instituição, a qual ordenastes para a propagação do gênero humano; assim aqueles unidos por vossa autoria, sejam servidos por vosso auxílio. Por Cristo Senhor nosso. R. Amém.

\end{multicols}

\textit{Em seguida, o celebrante, voltando-se para os esposos, eleva as mãos e as
estende sobre suas cabeças e diz:}


\begin{multicols}{2}
\noindent Benedícat vos Omnípotens Dóminus, ut cor vestrum sincéro amóre cópulet nexu perpétuo.
\\ \textbf{R. Amen.}
\\
\\ Floreátis rerum præséntium cópiis;  fructificétis decénter in fíliis et filiábus; gaudeátis perénniter cum fidélibus et amícis.
\\ \textbf{R. Amen.}
\\
\\ Tríbuat vobis Dóminus bona perénnia, præséntia per témpora felíciter dilatáta, et his cunctis gáudia sempitérna.
\\ \textbf{R. Amen.}
\\ 
\\ Ita Dóminus noster corda et córpora vestra bene$\dagger$dictióne circúmfluent in saécula sæculórum, quátenus post cursum vitæ, perveníre mereámini ad regna coelórum.
\\ \textbf{R. Amen.}
\\
\\Quod ipse præstáre dignétur, qui cum Patre et Sancto Spíritu vivit et regnat in saécula sæculórum.
\\ \textbf{R. Amen.}
\\
\\ Abençoe-vos o Senhor Onipotente vos, para que vossos corações se unam por laços de sincero amor. 
\\ R. Amém.
\\
\\ Que floresçais na abundância das coisas presentes; frutifiqueis, como convém, em filhos e filhas e vos alegreis perenemente com amigos fieis.
\\ R. Amém
\\
\\ Conceda-vos o Senhor bens perenais, no tempo presente, por uma copiosa felicidade e por cima disto tudo, os gáudios eternos.
\\R. Amém.
\\
\\Assim, o Senhor nosso cercará vossos corações e corpos de bên$\dagger$çãos pelos séculos dos séculos, até que que, após o curso desta vida, mereçais chegar ao reino dos céus.
\\R. Amém
\\
\\Que isto Se digne prestar-vos Aquele que, com o Pai e o Espírito Santo vive e reina nos séculos dos séculos.
\\R. Amém.
\end{multicols}
