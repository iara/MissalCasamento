\begin{center}
	\textbf{DESPEDIDA}
\end{center}
\begin{flushleft}
\textit{O celebrante volta ao meio do altar, beija-o, e, voltando-se para os fiéis saúda-os:}
\end{flushleft}

\begin{multicols}{2}
	\noindent Dóminus vobíscum.
	\\ \textbf{R. Et cum spíritu tuo.}
	\\ Ite, Missa est.
	\\ \textbf{R. Deo grátias.}
	\\O Senhor seja convosco.
	\\R. E com o teu espírito
	\\ Em boa hora vos ide.
	\\ R. R. Graças a Deus.
\end{multicols}

\textit{Voltando-se para o altar, recita a seguinte oração.}


\begin{multicols}{2}
	\noindent Pláceat tibi, sancta Trínitas, obséquium
	servitútis meæ: et præsta, ut sacrifícium
	quod óculis tuæ maiestátis indígnus
	óbtuli, tibi sit acceptábile, mihique, et
	ómnibus pro quibus illud óbtuli, sit, te
	miserante, propitiábile. Per Christum
	Dóminum nostrum. Amen.
	\\
	\\ Seja-vos agradável, ó Trindade santa, a
	oferta de minha servidão, a fim de que
	este sacrifício que, embora indigno aos
	olhos de vossa Majestade, vos ofereci,
	seja aceito por Vós, e por vossa
	misericórdia, seja propiciatório para
	mim e para todos aqueles por quem
	ofereci. Por Cristo Jesus Nosso Senhor.
	Amém.
\end{multicols}

\textit{Beija o altar, volta-se para a assistência, e dá a bênção, dizendo:}

\begin{multicols}{2}
	\noindent Benedicat vos omnípotens Deus: Pater, et
	Filius, $\dagger$ et Spíritus Sanctus.
	\\ \textbf{R. Amen.}
	\\ Abençoe-vos o Deus onipotente, Pai, e
	Filho, $\dagger$ e Espírito Santo.
	\\ R. Amém.
\end{multicols}

\begin{center}
	\textbf{ÚLTIMO EVANGELHO}
\end{center}

\begin{multicols}{2}
	\noindent Dóminus vobíscum.
	\\ \textbf{R. Et cum spíritu tuo.}
	\\  $\dagger$ Inítium sancti Evangélii secundum
	João Joannem.
	\\ \textbf{R. Glória tibi, Dómine.}
	\\O Senhor seja convosco.
	\\R. E com o teu espírito
	\\ $\dagger$ Início do santo Evangelho segundo São João.
	\\ R. Glória a Vós Senhor.
\end{multicols}

\begin{multicols}{2}
\noindent In princípio erat Verbum et Verbum erat apud Deum, et Deus erat Verbum. Hoc erat in princípio apud Deum. Omnia per ipsum facta sunt, et sine ipso factum est nihil quod factum est; in ipso vita erat, et vita erat lux hóminum; et lux in ténebris lucet, et ténebræ eam non comprehendérunt. 
\\ Fuit homo missus a Deo cui nomen erat Joánnes. Hic venit in testimónium, ut testimónium perhibéret de lúmine, ut omnes créderent per illum. Non erat ille lux, sed ut testimónium perhibéret de lúmine. 
\\Erat lux vera quæ illúminat omnem hóminem veniéntem in hunc mundum. In mundo erat, et mundus per ipsum factus est et mundus eum non cognóvit. In própria venit, et sui eum non recepérunt. Quotquot autem recepérunt eum, dedit eis potestátem fílios Dei fíeri; his qui credunt in nómine ejus, qui non ex sanguínibus, neque ex voluntáte carnis, neque ex voluntáte viri, sed ex Deo nati sunt. \textbf{(ajoelha-se)} ET VERBUM CARO FACTUM EST, et habitávit in nobis: et vídimus glóriam ejus, glóriam quasi Unigéniti a Patre, plenum grátiæ et veritátis.
\\ \textbf{R. Deo grátias.} 
\\
\\ No princípio era o Verbo, e o Verbo estava em Deus, e o Verbo era Deus. Estava Ele no princípio com Deus. Tudo por Ele foi feito, e nada de quanto se fez foi feito sem Ele. N’Ele estava a vida, e a vida era a luz dos homens; e a luz brilha nas trevas, e as trevas não a receberam. 
\\Houve um homem enviado por Deus, chamado João, o qual veio como testemunho, para dar testemunho da luz, a fim de que todos acreditassem por via dele. Não era ele a luz, mas veio para dar testemunho da luz. 
\\ Era (o Verbo) a luz verdadeira que ilumina todo o homem que vem a este mundo. Estava no mundo, e o mundo foi feito por Ele, e o mundo não O reconheceu. Veio para o que era seu, e os seus não O receberam. A todos, porém, quantos O receberam, deu Ele o poder de se tornarem filhos de Deus, quer dizer, àqueles que crêem no seu nome, que nem do sangue, nem do desejo da carne, nem da vontade do homem, mas só de Deus nasceram. (ajoelha-se) E O VERBO SE FEZ CARNE e veio habitar entre nós; e nós vimos a sua glória, glória do Filho Unigênito do Pai, cheio de graça e verdade.
\\R. Graças a Deus.

\end{multicols}