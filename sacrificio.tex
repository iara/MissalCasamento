\begin{center}
	\textbf{SEGUNDA PARTE DA MISSA: SACRIFÍCIO}
\end{center}

\begin{multicols}{2}
	\noindent Dóminus vobíscum.
	\\ \textbf{R. Et cum spíritu tuo.}
	\\O Senhor seja convosco.
	\\R. E com o teu espírito
\end{multicols}

\begin{flushleft}
	\textit{Durante o Ofertório, os fiéis ficam sentados. O celebrante inicia esta parte da missa com uma pequena oração, diferente para cada missa.}
\end{flushleft}

\begin{center}
	Salmo 30, 15-16
\end{center}
\begin{multicols}{2}
	\noindent In te sperávi, Domine: dixi: Tu es Deus meus: in mánibus tuis témpora mea 
	\\ Em Vós, Senhor, esperei e disse: Sois meu Deus, meu destino está em Vós 
\end{multicols}


\begin{multicols}{2}
	\noindent Súscipe, sancte Pater, omnípotens ætérne Deus, hanc immaculátam hóstiam, quam ego indígnus fámulus tuus óffero tibi, Deo meo, vivo et vero, pro innumerabílibus peccátis, et offensiónibus, et negligéntiis meis, et pro ómnibus circumstántibus, sed et pro ómnibus fidélibus christiánis vivis atque defúnctis: ut mihi, et illis profíciat ad salútem in vitam ætérnam. Amen. 
	\\Recebei, Pai santo, Deus onipotente e eterno, esta hóstia imaculada, que eu, vosso indigno servo, Vos ofereço a Vós, meu Deus vivo e verdadeiro, pelos meus inumeráveis pecados, ofensas e negligências, por todos os que estão aqui presentes e por todos os fiéis, vivos e defuntos, para que tanto a mim como a eles aproveita para a salvação e vida eterna Amém.
\end{multicols}

\begin{flushleft}
	\textit{Prepara o cálice, deitando vinho e água que benze dizendo:}
\end{flushleft}


\begin{multicols}{2}
	\noindent Deus, $\dagger$ qui humánæ substántiæ dignitátem mirabíliter condidísti et mirabílius reformásti: da nobis, per hujus aquæ et vini mystérium, ejus divinitátis esse consórtes, qui humanitátis nostræ fíeri dignátus est párticeps, Jesus Christus Fílius tuus, Dóminus noster: Qui tecum vivit et regnat in unitáte Spiritus Sancti Deus: per ómnia sáecula sæculórum. Amen. 
	\\
	\\Ó Deus, $\dagger$ que de modo maravilhoso criastes em sua dignidade a natureza humana e de modo mais maravilhoso ainda a reformastes, concedei-nos, pelo mistério desta água e vinho, sejamos participantes da divindade d’Aquele que se dignou partilhar da nossa humanidade, Jesus Cristo vosso Filho e Senhor nosso, que, sendo Deus, convosco vive e reina na unidade do Espírito Santo por todos os séculos dos séculos. Amém
\end{multicols}

\begin{flushleft}
	\textit{Oferece o cálice.}
\end{flushleft}

\begin{multicols}{2}
	\noindent Offérimus tibi, Dómine, cálicem salutáris, tuam deprecántes cleméntiam: ut in conspéctu divínæ majestátis tuæ, pro nostra et totíus mundi salúte, cum odóre suavitátis ascéndat. Amen. 
	\\ Nós Vos oferecemos, Senhor, o cálix da salvação, e da vossa clemência imploramos que ele se eleve até à presença da vossa divina majestade, qual suave odor, para salvação nossa e de todo o mundo. Amém.
\end{multicols}

\begin{flushleft}
	\textit{Depois acrescenta:}
\end{flushleft}

\begin{multicols}{2}
\noindent In spíritu humilitátis et in ánimo contríto suscipiámur a te, Dómine: et sic fiat sacrifícium nostrum in conspéctu tuo hódie, ut pláceat tibi, Dómine Deus. Veni, Sanctificátor, omnípotens ætérne Deus: et bénedic + hoc sacrifícium, tuo sancto nómini præparátum. 
\\
\\ Com o espírito humilhado e coração contrito, sejamos por Vós acolhidos, Senhor; e que este nosso sacrifício se realize hoje na vossa presença por forma a merecer o vosso agrado, Senhor nosso Deus.
Vinde, Santificador, Deus onipotente e eterno, e abençoai + este sacrifício preparado para o vosso santo nome.
\end{multicols}

\begin{flushleft}
	\textit{Depois, lava os dedos dizendo o Salmo 25.}
\end{flushleft}


\begin{multicols}{2}
	\noindent Lavabo inter innocéntes manus meas: et circúmdabo altáre tuum, Dómine. 
	Ut áudiam vocem laudis, et enárrem univérsa mirabília tua. 
	Dómine, diléxi decórem domus tua, et locum habitatiónis glóriæ tuæ.
	Ne perdas cum ímpiis, Deus, ánimam meam, et cum viris sánguinum vitam meam: 
	In quorum mánibus iniquitátes sunt: déxtera eórum repléta est munéribus. 
	Ego autem in innocéntia mea ingréssus sum: rédime me, et miserére mei. 
	Pes meus stetit in dirécto: in ecclésiis benedícam te, Dómine. 
	Glória Patri, et Fílio, et Spirítui Sancto. Sicut erat in princípio, et nunc, et semper: et in sáecula sæculórum. Amen. 
	\\ Lavo na inocência as minhas mãos, e acerco-me, Senhor, do vosso altar,
	Para fazer ouvir os vossos louvores, e apregoar todas as vossas maravilhas.
	Amo, Senhor, a beleza da vossa casa, e o lugar em que repousa a vossa glória.
	Não deixeis, ó Deus, que minha alma se perca com os pecadores, nem a minha vida com os homens sanguinários;
	Eles que têm as mãos manchadas de iniqüidade, e a dextra de peitas repleta.
	Eu, pelo contrário, conduzo-me pelas sendas da inocência; livrai-me, Senhor, e compadecei-Vos de mim.
	Os meus pés andam pelo caminho da retidão; nas assembléias Vos bendirei, Senhor.
	Glória ao Pai, ao Filho e ao Espírito Santo. Assim como era no princípio, agora e sempre, por todos os séculos dos séculos, Amém.
\end{multicols}

\begin{center}
	\MakeUppercase{Oração à Santíssima Trindade} 
\end{center}
\textit{Regressa o sacerdote ao meio do altar, e, inclinando-se profundamente, diz:}
\begin{multicols}{2}
	\noindent Súscipe, sancta Trínitas, hanc oblatiónem, quam tibi offérimus ob memóriam passiónis, resurrectiónis, et ascensiónis Jesu Christi, Dómini nostri, et in honórem beátæ Maríæ semper Vírginis, et beáti Joannis Baptistæ, et sanctórum apostolórum Petri et Pauli, et istórum, et ómnium sanctórum: ut illis profíciat ad honórem, nobis autem ad salútem: et illi pro nobis intercédere dignéntur in cælis, quorum memóriam ágimus in terris. Per eúmdem Christum Dóminum nostrum. Amen. 
    \\ Recebei, Trindade Santa, esta oblação, que Vos oferecemos em memória da Paixão, Ressurreição e Ascensão de Jesus Cristo nosso Senhor, e em honra da bem-aventurada sempre Virgem Maria, de S.João Batista, dos Santos Apóstolos Pedro e Paulo, destes (cujas relíquias aqui estão) e de todos os Santos; seja para honra deles e salvação nossa, e por nós se dignem intereceder no céu aqueles cuja memória celebramos na terra. Pelo mesmo Jesus Cristo Senhor nosso. Amém.
\end{multicols}

\begin{center}
	ORATE FRATES
\end{center}

\begin{flushleft}
\textit{O sacerdote volta-se para o povo:}
\end{flushleft}

\begin{multicols}{2}
	\noindent Oráte fratres, ut meum ac vestrum sacrifícium acceptábile fiat apud Deum Patrem omnipoténtem.
	\\ \textbf{R. Suscípiat Dóminus sacrifícium de mánibus tuis  ad laudem et glóriam nóminis sui, ad utilitátem quoque nostram, totiúsque Ecclésiæ suæ sanctæ.}
	\\ Orai, irmãos, para que este sacrifício, meu e vosso, seja aceite de Deus Pai onipotente. 
	\\ R. Receba o Senhor das tuas mãos este sacrifício para louvor e glória do seu nome e para bem nosso e de toda a sua santa Igreja.
	Amém.
\end{multicols}

\begin{flushleft}
	\textit{Em seguida lê a Secreta. }
\end{flushleft}
\begin{center}
	SECRETA
\end{center}

\begin{multicols}{2}
	\noindent Súscipe, quáesumus, Dómine, pro sacra connúbii lege munus oblátum: et, cujus largítor es óperis, esto dispósitor. Per Dóminum nostrum\textit{ per ómnia saécula saeculorum.}
	\\ \textbf{R. Amen.}
	\\
	\\
	\\
	\\ Dignai-Vos, Senhor, aceitar a oferta por este sacro Matrimônio; e, pois sois seu dispenseiro, sede também ordenador. Por Nosso Senhor Jesus Cristo, vosso Filho, que, sendo Deus, convosco vive e reina na unidade do Espírito Santo  \textit{por todos os séculos dos séculos. }
	\\ R. Amém.
\end{multicols}

